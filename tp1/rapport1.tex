\documentclass[a4paper,10pt]{article}
\usepackage[utf8]{inputenc}

%opening
\title{Eléments de géométrie projective et calibration de caméra}
\author{Elliot Vanegue}

\begin{document}

\maketitle

\section{Introduction}
objectif : determiner les 11 degrès de liberté par la méthode de zhang

\section{Étude de la publication de Zhang}
La méthode de Zhang admet que la mire est en Z=0 dans le repère monde. Cela
implique que le paramètre de rotation r3 de la matrice extrinsèque sera supprimer dans la formule
\textit{mettre la 1er formule du paragraphe 2.2}. Cela va permettre de déterminer une homographie contenant les autres
paramètre extrinsèque.
Cependant, une homographie possède huit degrès de liberté, alors que la matrice qui a été déterminé
n'en possède que six. La méthode de Zhang va donc rajouter à cette homographie des paramètres provenant
de la matrice intrinsèque.

\end{document}
