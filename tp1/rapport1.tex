\documentclass[a4paper,10pt]{article}
\usepackage[utf8]{inputenc}
\usepackage{amsmath}
\usepackage{amssymb}

%opening
\title{Eléments de géométrie projective et calibration de caméra}
\author{Elliot Vanegue}

\begin{document}

\maketitle

\section{Introduction}
Lors de ce TP, nous allons chercher à déterminer les paramètres extrinsèque et 
intrinsèque d'une caméra en utilisant la méthode de calibration de Zhang.
Cette méthode prend plusieurs images donc chaque angle de vue est différent, mais
où la mire se place toujours en Z=0. Cela va permettre de réduire le nombre de paramètre
à calculer.

\section{Étude de la publication de Zhang}
La méthode de Zhang admet que la mire est en Z=0 dans le repère monde. Cela
implique que le paramètre de rotation r3 de la matrice extrinsèque sera supprimer dans la matrice extrinsèque.
Ce qui nous permet d'avoir le calcul suivant : 
$$s\begin{pmatrix}u\\v\\1\end{pmatrix} = A\begin{pmatrix}r_1&r_2&t\end{pmatrix}\begin{pmatrix}x\\y\\1\end{pmatrix}$$

La méthode de Zhang détermine ensuite une homographie à partir du vecteur du point du modèle 3D avec celui
du modèle 2D. Cette homographie va permettre de construire une matrice de contrainte permettant de 
calculer les paramètres intrinsèques de la caméra. Cette matrice de contrainte va se calculer de la manière suivante :
$$v_ij = [h_{i1}*h_{j1}, h_{i1}*h_{j2} + h_{i2}*h_{j1}, h_{i2}*h_{j2}, h_{i3}*h_{j1} + h_{i1}*h_{j3}, h_{i3}*h_{j2} + h_{i2}*h_{j3}, h_{i3}*h_{j3}]$$

%de la forme $$2nx6$$ où n est le nombre d'image traité
Cela va permettre de déterminer une homographie contenant les autres paramètres extrinsèques.
Cependant, une homographie possède huit degrès de liberté, alors que la matrice qui a été déterminé
n'en possède que six. La méthode de Zhang va donc rajouter à cette homographie des paramètres provenant
de la matrice intrinsèque.

\end{document}
