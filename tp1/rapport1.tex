\documentclass[a4paper,10pt]{article}
\usepackage[utf8]{inputenc}
\usepackage{amsmath}
\usepackage{amssymb}
\usepackage{amsfonts}

%opening
\title{Eléments de géométrie projective et calibration de caméra}
\author{Elliot Vanegue}

\begin{document}

\maketitle
\newpage
\renewcommand{\contentsname}{Sommaire}
\tableofcontents
\newpage

\section{Introduction}
Lors de ce TP, nous allons chercher à déterminer les paramètres extrinsèque et 
intrinsèque d'une caméra en utilisant la méthode de calibration de Zhang.
Cette méthode prend plusieurs images donc chaque angle de vue est différent, mais
où la mire se place toujours en Z=0. Cela va permettre de réduire le nombre de paramètre
à calculer.

\section{Étude de la publication de Zhang}

La méthode de Zhang permet de déterminer les paramètres de calibration d'une caméra
à partir de plusieurs points de vue. Elle part du calcule suivant :
\begin{equation}
 s\begin{pmatrix}u\\v\\1\end{pmatrix} = A\begin{pmatrix}r_1&r_2&r_3&t\end{pmatrix}\begin{pmatrix}x\\y\\0\\1\end{pmatrix}
\end{equation}
La méthode de Zhang admet que la mire est en Z=0 dans le repère monde. Cela
implique que le paramètre de rotation r3 de la matrice extrinsèque sera supprimer.
Cette méthode utilise une matrice B qui est égale à $A^{-T}A^{-1}$ afin de déterminer la valeur
des paramètres de la matrice intrinsèque. Pour calculer cette matrice B il faut utiliser le
calcul suivant : 
\begin{equation}
 h^T_iBh_j=v^T_{ij}b
\end{equation}
Ici la valeur de v représente des contraintes qui dépendent du nombre d'image utilisé pour 
la calibration.
Une fois que la matrice intrinsèque est calculé, il
est possible de retrouver les paramètres extrinsèques à partir de la formule (1).

\section{Calcul de la matrice de contraintes V}
La matrice de contrainte V utilise la formule suivante :
\begin{multline}
 v_ij = [h_{i1}*h_{j1}, h_{i1}*h_{j2} + h_{i2}*h_{j1}, h_{i2}*h_{j2},  \\
 h_{i3}*h_{j1} + h_{i1}*h_{j3}, h_{i3}*h_{j2} + h_{i2}*h_{j3}, h_{i3}*h_{j3}]
\end{multline}

Si on prend l'équation (2) on voit que la valeur manquante est b. Pour calculer cette valeur il faut
utiliser la formule suivante :
\begin{equation}
 \begin{pmatrix}v^T_{12}\\(v_{11}-v_{22})^T\end{pmatrix}b=0
\end{equation}
Lorsque cette formule est utilisé avec plus de trois images, la valeur de b devient unique et on obtient
alors $$Vb=0$$ V étant calculé avec la formule (3). Ainsi b est de la forme $[B_{11}, B_{12}, B_{22}, B_{13}, B_{23}, B_{33}]$.

\section{Calcul de la matrice intrinsèque}
Maintenant que la matrice b est calculé, il suffit de l'utiliser afin de déterminer les différents 
paramètres de la matrice intrinsèque avec les calcules suivants : 

\begin{align}
  &v0     = \frac{B_{12}*B_{13} - B_{11}*B_{23}}{B_{11}*B_{22} - B^2_{12}}\\
  &\lambda = b(6) - \frac{B^2_{13} + v0 * (B_{12} * B_{13} - B_{11} * B_{23})}{B_{11}}\\
  &\alpha  = \sqrt{\frac{\lambda}{B_{11}}}\\
  &\beta   = \sqrt{\frac{\lambda*B_{11}}{B_{11}*B_{22} - B^2_{12}}}\\
  &\gamma  = \frac{-B_{12}*\alpha^2*\beta}{\lambda}\\
  &u0     = \frac{\gamma *v0}{\beta} - \frac{B_{13}*\alpha^2}{\lambda}\\
\end{align}

On obtient ainsi les valeurs suivantes pour notre cas : 
$$\begin{pmatrix} 3498 & -3,1 & 336,8\\0 & 3503 & 220,1\\ 0 & 0 & 1\end{pmatrix}$$
On peut voir que ces résultats sont assez proche des résultats attendu avec néanmoins
une légère variation sur les facteurs d'échelle.

\section{Calcul de la matrice extrinsèque}
Il est maintenant possible de déterminer les paramètres extrinsèque à partir de la formule (1)
étant donné que l'homographie est connu ainsi des paramètres intrinsèques. Il faut donc utiliser
les formules suivantes pour calculer la matrice extrinsèque : 

\begin{align}
 &\lambda = \frac{1}{||A^{-1}h_1||}\\
 &r_1=\lambda A^{-1}h_1\\
 &r_2=\lambda A^{-1}h_2\\
 &r_3=r_1*r_2\\
 &t=\lambda A^{-1}h_3
\end{align}

Ce qui nous donne les résultats suivants pour la matrice extrinsèque :\\ 

\begin{center}
\begin{tabular}{|c|c|}
 \hline
 numéro de l'image & résultat\\
 \hline
 1 & $\begin{pmatrix} 1 & 0 & 0 & -49\\0 & 1 & 0 & 55\\ 0 & 0 & 0 & 9854\end{pmatrix}$\\
 \hline
 2 & $\begin{pmatrix} 1 & 0 & 0 & -46\\0 & 1 & 0 & 44\\ -1 & 0 & 0 & 7905\end{pmatrix}$\\
 \hline
 3 & $\begin{pmatrix} 1 & 0 & 0 & -44\\0 & 1 & 0 & 49\\ 0 & 0 & 0 & 8870\end{pmatrix}$\\
 \hline
 4 & $\begin{pmatrix} 1 & 0 & 0 & -144\\0 & 1 & 0 & 42\\ 0 & -1 & 0 & 8872\end{pmatrix}$\\
 \hline
\end{tabular}
\end{center}
On voit que les résultat des translations de chacune des images sont assez proche des résultats
attendu. En revanche, les résultats pour les rotations ne sont pas correctes, car les rotations
ne sont par effectué sur le bonne axe.

\section{Calcul de la distance focal}
Il est possible de calculer la distance focal en décomposant la formule (1).
\begin{equation}
 s\begin{pmatrix}u\\v\\1\end{pmatrix} = \begin{pmatrix}k_u&s_{uv}&u_0\\0&k_v&v_0\\0&0&1\end{pmatrix}\begin{pmatrix}f&0&0\\0&f&0\\0&0&1\end{pmatrix}
					\begin{pmatrix}r_1&r_2&r_3&t\end{pmatrix}\begin{pmatrix}x\\y\\0\\1\end{pmatrix}
\end{equation}
Si on nomme K la matrice $\begin{pmatrix}k_u&s_{uv}&u_0\\0&k_v&v_0\\0&0&1\end{pmatrix}\begin{pmatrix}r_1&r_2&t\end{pmatrix}$, il est possible de 
remplacer la matrice B dans la formule (2) par $K^{-T}K^{-1}$. 
Il serait alors possible de calculer la matrice focal grâce aux formules :
\begin{align}
 &\lambda = \frac{1}{||K^{-1}h_1||}\\
 &f_1=\lambda K^{-1}h_1\\
 &f_2=\lambda K^{-1}h_2\\
 &f_3=f_1*f_2\\
\end{align}

\end{document}
