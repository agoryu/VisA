\documentclass[a4paper,10pt]{article}
\usepackage[utf8]{inputenc}
\usepackage{amsmath}
\usepackage{amssymb}
\usepackage{amsfonts}

% define our color
\usepackage{xcolor}
% insert code
\usepackage{listings}

% code color
\definecolor{ligthyellow}{RGB}{250,247,220}
\definecolor{darkblue}{RGB}{5,10,85}
\definecolor{ligthblue}{RGB}{1,147,128}
\definecolor{darkgreen}{RGB}{8,120,51}
\definecolor{darkred}{RGB}{160,0,0}

\lstset{
    language=Scilab,
    captionpos=b,
    extendedchars=true,
    frame=lines,
    numbers=left,
    numberstyle=\tiny,
    numbersep=5pt,
    keepspaces=true,
    breaklines=true,
    showspaces=false,
    showstringspaces=false,
    breakatwhitespace=false,
    stepnumber=1,
    showtabs=false,
    tabsize=3,
    basicstyle=\small\ttfamily,
    backgroundcolor=\color{ligthyellow},
    keywordstyle=\color{ligthblue},
    morekeywords={include, printf, uchar},
    identifierstyle=\color{darkblue},
    commentstyle=\color{darkgreen},
    stringstyle=\color{darkred},
}

%opening
\title{Eléments de géométrie projective et calibration de caméra}
\author{Elliot Vanegue}

\begin{document}

\maketitle
% \newpage
% \renewcommand{\contentsname}{Sommaire}
% \tableofcontents
% \newpage

\section{Introduction}
Lors de ce TP, nous allons chercher à déterminer les paramètres extrinsèques et 
intrinsèques d'une caméra en utilisant la méthode de calibration de Zhang.
Cette méthode prend plusieurs images, donc chaque angle de vue est différent, mais
où la mire se place toujours en Z=0. Cela va permettre de réduire le nombre de paramètres
à calculer.

\section{Étude de la publication de Zhang}

La méthode de Zhang utilise une matrice B qui est égale à $A^{-T}A^{-1}$, où A est la matrice
intrinsèque. Il faut donc déterminer la valeur de la matrice B afin de pouvoir ensuite 
calculer les paramètres intrinsèques. Pour calculer cette matrice B, il faut utiliser le
calcul suivant : 
\begin{equation}
 h^T_iBh_j=v^T_{ij}b
\end{equation}
Ici la valeur de v représente des contraintes qui dépendent du nombre d'images utilisées pour 
la calibration. Les valeurs $h^T_i$ et $h_j$ sont connues et représentent des valeurs présentes
dans l'homographie composée des matrices $\begin{pmatrix}u\\v\\1\end{pmatrix}$ et $\begin{pmatrix}x\\y\\1\end{pmatrix}$.
Une fois que la matrice intrinsèque est calculée, il
est possible de retrouver les paramètres extrinsèques à partir de la formule :
\begin{equation}
 s\begin{pmatrix}u\\v\\1\end{pmatrix} = A\begin{pmatrix}r_1&r_2&r_3&t\end{pmatrix}\begin{pmatrix}x\\y\\z\\1\end{pmatrix}
\end{equation}
Cette méthode admet que la mire est plane, ce qui signifie que Z=0 dans le repère monde. Cela
implique que le paramètre de rotation r3 de la matrice extrinsèque sera nul. Il est donc
possible de supprimer ce paramètre.\\

\section{Calcul de la matrice de contraintes V}
La matrice de contraintes V utilise la formule suivante :
\begin{multline}
 v_ij = [h_{i1}*h_{j1}, h_{i1}*h_{j2} + h_{i2}*h_{j1}, h_{i2}*h_{j2},  \\
 h_{i3}*h_{j1} + h_{i1}*h_{j3}, h_{i3}*h_{j2} + h_{i2}*h_{j3}, h_{i3}*h_{j3}]
\end{multline}

Si on prend l'équation (1), on voit que la valeur manquante est b. Pour calculer cette valeur, il faut
utiliser la formule suivante :
\begin{equation}
 \begin{pmatrix}v^T_{12}\\(v_{11}-v_{22})^T\end{pmatrix}b=0
\end{equation}
Lorsque cette formule est utilisée avec plus de trois images, la valeur de b devient unique et on obtient
alors $Vb=0$, V étant calculé avec la formule (3). Ainsi b est de la forme $[B_{11}, B_{12}, B_{22}, B_{13}, B_{23}, B_{33}]$.

\section{Calcul de la matrice intrinsèque}
Maintenant que la matrice b est calculée, il suffit de l'utiliser afin de déterminer les différents 
paramètres de la matrice intrinsèque avec les calculs suivants : 

\begin{align}
  &v0     = \frac{B_{12}*B_{13} - B_{11}*B_{23}}{B_{11}*B_{22} - B^2_{12}}\\
  &\lambda = b(6) - \frac{B^2_{13} + v0 * (B_{12} * B_{13} - B_{11} * B_{23})}{B_{11}}\\
  &\alpha  = \sqrt{\frac{\lambda}{B_{11}}}\\
  &\beta   = \sqrt{\frac{\lambda*B_{11}}{B_{11}*B_{22} - B^2_{12}}}\\
  &\gamma  = \frac{-B_{12}*\alpha^2*\beta}{\lambda}\\
  &u0     = \frac{\gamma *v0}{\beta} - \frac{B_{13}*\alpha^2}{\lambda}\\
\end{align}

On obtient ainsi les valeurs suivantes pour notre cas : 
$$\begin{pmatrix} 3498 & -3,1 & 336,8\\0 & 3503 & 220,1\\ 0 & 0 & 1\end{pmatrix}$$
On peut voir que ces résultats sont assez proches des résultats attendus avec néanmoins
une légère variation sur les facteurs d'échelle.

\section{Calcul de la matrice extrinsèque}
Il est maintenant possible de déterminer les paramètres extrinsèques à partir de la formule (2),
étant donné que l'homographie est connue, ainsi que les paramètres intrinsèques. Il faut donc utiliser
les formules suivantes pour calculer la matrice extrinsèque : 

\begin{align}
 &\lambda = \frac{1}{||A^{-1}h_1||}\\
 &r_1=\lambda A^{-1}h_1\\
 &r_2=\lambda A^{-1}h_2\\
 &r_3=r_1*r_2\\
 &t=\lambda A^{-1}h_3
\end{align}

Ce qui nous donne les résultats suivants pour la matrice extrinsèque :\\ 

\begin{center}
\begin{tabular}{|c|c|}
 \hline
 numéro de l'image & résultat\\
 \hline
 1 & $\begin{pmatrix} 1 & 0 & 0 & -49\\0 & 1 & 0 & 55\\ 0 & 0 & 0 & 9854\end{pmatrix}$\\
 \hline
 2 & $\begin{pmatrix} 0.7 & 0 & 0 & -46\\0 & 1 & 0 & 44\\ -0.7 & 0 & 0 & 7905\end{pmatrix}$\\
 \hline
 3 & $\begin{pmatrix} 1 & 0 & 0 & -44\\0 & 1 & 0 & 49\\ 0 & 0 & 0 & 8870\end{pmatrix}$\\
 \hline
 4 & $\begin{pmatrix} 1 & 0 & 0 & -144\\0 & 0.7 & 0 & 42\\ 0 & -0.7 & 0 & 8872\end{pmatrix}$\\
 \hline
\end{tabular}
\end{center}
On voit que les résultats des translations de chacune des images sont assez proches des résultats
attendus. 
%En revanche, les résultats pour les rotations ne sont pas correctes, car les rotations
%ne sont par effectué sur le bonne axe.

\section{Calcul de la distance focale}
Il est possible de calculer la distance focale en décomposant la formule (2).
\begin{equation}
 s\begin{pmatrix}u\\v\\1\end{pmatrix} = \begin{pmatrix}f&0&0\\0&f&0\\0&0&1\end{pmatrix}\begin{pmatrix}k_u&s_{uv}&u_0\\0&k_v&v_0\\0&0&1\end{pmatrix}
					\begin{pmatrix}r_1&r_2&r_3&t\end{pmatrix}\begin{pmatrix}x\\y\\0\\1\end{pmatrix}
\end{equation}
Si on nomme K la matrice $\begin{pmatrix}k_u&s_{uv}&u_0\\0&k_v&v_0\\0&0&1\end{pmatrix}\begin{pmatrix}r_1&r_2&t\end{pmatrix}$, il est possible de 
remplacer la matrice B dans la formule (2) par $K^{-T}K^{-1}$. 
Il serait alors possible de calculer la matrice focale grâce aux formules :
\begin{align}
 &\lambda = \frac{1}{||K^{-1}h_1||}\\
 &f_1=\lambda K^{-1}h_1\\
 &f_2=\lambda K^{-1}h_2\\
 &f_3=f_1*f_2\\
\end{align}

\section{Conclusion}
La méthode de Zhang est donc une méthode très efficace pour déterminer l'ensemble des paramètres
de calibration d'une caméra. Les résultats sont très proches de ceux attendus. Cette méthode
permet également de déterminer la focale de la caméra si tous les paramètres sont connus.

\section{Annexes}
\begin{lstlisting}[caption=Fonctions utilise pour la methode de Zhang]
// ---------------------------------------------
/// \brief Calcule un terme de contrainte a partir d'une homographie.
///
/// \param H: matrice 3*3 definissant l'homographie.
/// \param i: premiere colonne.
/// \param j: deuxieme colonne.
/// \return vecteur definissant le terme de contrainte.
// --------------------------------------------
function v = ZhangConstraintTerm(H, i, j)
  // A modifier!
  //v = rand(1, 6);
  v = [H(1,i)*H(1,j), H(1,i)*H(2,j) + H(2,i)*H(1,j), H(2,i)*H(2,j),...
        H(3,i)*H(1,j) + H(1,i)*H(3,j), H(3,i)*H(2,j) + H(2,i)*H(3,j), H(3,i)*H(3,j)];
endfunction

// ------------------------------------------
/// \brief Calcule deux equations de contrainte a partir d'une homographie
///
/// \param H: matrice 3*3 definissant l'homographie.
/// \return matrice 2*6 definissant les deux contraintes.
// -------------------------------------------
function v = ZhangConstraints(H)
  v = [ZhangConstraintTerm(H, 1, 2); ...
    ZhangConstraintTerm(H, 1, 1) - ZhangConstraintTerm(H, 2, 2)];
endfunction

// ------------------------------------------
/// \brief Calcule la matrice des parametres intrinseques.
///
/// \param b: vecteur resultant de l'optimisation de Zhang.
/// \return matrice 3*3 des parametres intrinseques.
// ------------------------------------------
function A = IntrinsicMatrix(b)
  // A modifier!
  //A = rand(3, 3);
  v0 = (b(2)*b(4) - b(1)*b(5)) / (b(1)*b(3) - b(2)*b(2));
  lambda = b(6) - (b(4) * b(4) + v0 * (b(2) * b(4) - b(1) * b(5))) / b(1);
  alpha = sqrt(lambda/b(1));
  bet = sqrt(lambda*b(1)/(b(1)*b(3) - b(2)*b(2)));
  gama = -b(2)*alpha*alpha*bet/lambda;
  u0 = gama*v0/bet - (b(4)*alpha*alpha)/lambda;
  
  A = [alpha, gama, u0; 0, bet, v0; 0, 0, 1];
endfunction

// ---------------------------------------------
/// \brief Calcule la matrice des parametres extrinseques.
///
/// \param iA: inverse de la matrice intrinseque.
/// \param H: matrice 3*3 definissant l'homographie.
/// \return matrice 3*4 des parametres extrinseques.
// ---------------------------------------------
function E = ExtrinsicMatrix(iA, H)
  // A modifier!
  //E = rand(3, 4);
  lambda = 1 / norm(iA*H(:,1));
  r1 = lambda * iA * H(:,1);
  r2 = lambda * iA * H(:,2);
  r3 = r1 .* r2;
  t = lambda * iA * H(:,3);
  E = [r1, r2, r3, t];
endfunction

\end{lstlisting}

\end{document}
