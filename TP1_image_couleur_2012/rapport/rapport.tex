\documentclass[a4paper,10pt]{article}
\usepackage[utf8]{inputenc}

%pour les equations
\usepackage{amsmath}
\usepackage{amssymb}
\usepackage{amsfonts}

%pour les images
\usepackage{graphicx}

% pour definir des couleurs
\usepackage{xcolor}

% pour inclure du code
\usepackage{listings}

% code color
\definecolor{ligthyellow}{RGB}{250,247,220}
\definecolor{darkblue}{RGB}{5,10,85}
\definecolor{ligthblue}{RGB}{1,147,128}
\definecolor{darkgreen}{RGB}{8,120,51}
\definecolor{darkred}{RGB}{160,0,0}

\lstset{
    language=C++,
    captionpos=b,
    extendedchars=true,
    frame=lines,
    numbers=left,
    numberstyle=\tiny,
    numbersep=5pt,
    keepspaces=true,
    breaklines=true,
    showspaces=false,
    showstringspaces=false,
    breakatwhitespace=false,
    stepnumber=1,
    showtabs=false,
    tabsize=3,
    basicstyle=\small\ttfamily,
    backgroundcolor=\color{ligthyellow},
    keywordstyle=\color{ligthblue},
    morekeywords={include, printf, uchar},
    identifierstyle=\color{darkblue},
    commentstyle=\color{darkgreen},
    stringstyle=\color{darkred},
}

%opening
\title{Image Couleur}
\author{Elliot Vanegue}

\begin{document}

\maketitle
\section{Introduction}

\section{Manipulation luminance}
Nous allons dans un premier comparer deux images dont l'information de luminance est différente.
Pour effectuer cette comparaison, nous allons devoir utiliser un autre espace colorimètrique qui
sait représenter cette information. Plusieurs espaces sont disponible comme le YUV ou le HSV.
Dans notre cas, nous allons prendre l'espace HSV dont la valeur (\enquote{value}) représente 
la luminance présente dans l'image.

%TODO ajout des images

On voit clairement que l'image qui est plus sombre ne possède pas de pixel dans le haut du conne
HSV.

2) Meilleur valeur -> 37 - 40


\end{document}
