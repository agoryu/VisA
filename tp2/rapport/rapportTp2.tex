\documentclass[a4paper,10pt]{article}
\usepackage[utf8]{inputenc}
\usepackage{amsmath}
\usepackage{amssymb}
\usepackage{amsfonts}

% define our color
\usepackage{xcolor}
% insert code
\usepackage{listings}

% code color
\definecolor{ligthyellow}{RGB}{250,247,220}
\definecolor{darkblue}{RGB}{5,10,85}
\definecolor{ligthblue}{RGB}{1,147,128}
\definecolor{darkgreen}{RGB}{8,120,51}
\definecolor{darkred}{RGB}{160,0,0}

\lstset{
    language=C++,
    captionpos=b,
    extendedchars=true,
    frame=lines,
    numbers=left,
    numberstyle=\tiny,
    numbersep=5pt,
    keepspaces=true,
    breaklines=true,
    showspaces=false,
    showstringspaces=false,
    breakatwhitespace=false,
    stepnumber=1,
    showtabs=false,
    tabsize=3,
    basicstyle=\small\ttfamily,
    backgroundcolor=\color{ligthyellow},
    keywordstyle=\color{ligthblue},
    morekeywords={include, printf, uchar},
    identifierstyle=\color{darkblue},
    commentstyle=\color{darkgreen},
    stringstyle=\color{darkred},
}

%opening
\title{Mise en correspondance stéréoscopique}
\author{Elliot Vanegue}

\begin{document}

\maketitle

\section{Introduction}
Lors de ce TP, nous allons reconstituer la troisième dimension d'une image
qui a été perdu suite à la projection de celle-ci. Pour cela, nous allons 
utiliser la stéréovision afin de reconstruire la troisième dimension de certains points à
partir de deux images dont les angles de vues sont différents.

\section{Calcul de la matrice fondamentale}
La matrice fondamentale contient l'ensemble des informations de la géométrie épipolaire.
Elle est constitué de sept degrès de liberté. Cette géométrie décrit les relations entre
différentes photographies du même objet. Pour la calculer, il faut utiliser la formule suivante :
\begin{equation}
 F=(P_2O_1)^xP_2P^+_1
 \label{fondamentale}
\end{equation}
Pour calculer F, on peut voir qu'il faut utiliser un produit vectoriel afin de 
pouvoir calculer la droite passant par les points de chaque image. Pour cela,
à partir d'un vecteur p, il faut construire la matrice suivante :
\begin{equation}
 p^x=\begin{pmatrix}
  0 & -p_z & p_y\\
  p_z & 0 & -p_x\\
  -p_y & p_x & 0
 \end{pmatrix}
 \label{vectoriel}
\end{equation}

Nous allons maintenant détailler la façon de calculer la matrice fondamentale afin
d'obtenir l'équation \eqref{fondamentale}. Tout d'abord, nous devons calculer
les matrices $P_1$ et $P_2$ qui sont respectivement la multiplication des matrices
intrinsèque et extrinsèque de l'image de gauche et de l'image de droite. Il faut
d'abord multiplier la matrice extrinsèque par la matrice $\begin{pmatrix} 1&0&0&0\\0&1&0&0\\0&0&1&0\end{pmatrix}$,
car la matrice extrinsèque est de la forme 4x3 alors que la matrice intrinsèque est de la forme
3x3. Ce qui nous donne le calcul suivant :
\begin{align}
 &P_1 = mLeftIntrinsic * \begin{pmatrix} 1&0&0&0\\0&1&0&0\\0&0&1&0\end{pmatrix} * mLeftExtrinsic\\
 &P_2 = mRightIntrinsic * \begin{pmatrix} 1&0&0&0\\0&1&0&0\\0&0&1&0\end{pmatrix} * mRightExtrinsic
 \label{matriceP}
\end{align}

Il ne reste plus qu'à déterminer la valeur de $O_1$ qui est un point sur l'image de gauche.
Pour calculer ces coordonnées, il faut inverser la matrice extrinsèque gauche et récupérer
les valeurs de la dernière colonne de cette matrice. Il reste alors à calculer la matrice
fondamentale avec le calcul \eqref{fondamentale}(voir \ref{Afondamentale}).

\section{Détermination d'équations de droite}
%Grâce au calcul de la droite fondamentale, il est possible de déterminer les équations de droite


\section{Annexes}
\subsection{Annexe A}
\label{AproduitVector}
\begin{lstlisting}[caption=Calcul produit vectoriel]
 Mat iviVectorProductMatrix(const Mat& v) {
     Mat mVectorProduct = (Mat_<double>(3,3) <<
             0.0, -v.at<double>(0,2), v.at<double>(0,1),
             v.at<double>(0,2), 0.0, -v.at<double>(0,0),
             -v.at<double>(0,1), v.at<double>(0,0), 0.0);
     return mVectorProduct;
 }
\end{lstlisting}

\subsection{Annexe B}
\label{Afondamentale}
\begin{lstlisting}[caption=Calcul matrice fondamentale]
 Mat iviFundamentalMatrix(const Mat& mLeftIntrinsic,
                         const Mat& mLeftExtrinsic,
                         const Mat& mRightIntrinsic,
                         const Mat& mRightExtrinsic) {
                         
    Mat mFundamental = Mat::eye(3, 3, CV_64F);
   
    Mat tmp = (Mat_<double>(3,4) <<
        1.0, 0.0, 0.0, 0.0,
        0.0, 1.0, 0.0, 0.0,
        0.0, 0.0, 1.0, 0.0
        );
    Mat P1 = mLeftIntrinsic * tmp * mLeftExtrinsic;
    Mat P2 = mRightIntrinsic * tmp * mRightExtrinsic;

    Mat O = mLeftExtrinsic.inv();
    Mat O1 = (Mat_<double>(4,1) <<
        O.at<double>(3),
        O.at<double>(7),
        O.at<double>(11),
        O.at<double>(15)
        );

    Mat Hpi = P2 * P1.inv(DECOMP_SVD);
    mFundamental = iviVectorProductMatrix(P2*O1) * Hpi;

    return mFundamental;
}
\end{lstlisting}

\end{document}
