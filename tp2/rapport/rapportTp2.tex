\documentclass[a4paper,10pt]{article}
\usepackage[utf8]{inputenc}
\usepackage{amsmath}
\usepackage{amssymb}
\usepackage{amsfonts}

% define our color
\usepackage{xcolor}
% insert code
\usepackage{listings}

% code color
\definecolor{ligthyellow}{RGB}{250,247,220}
\definecolor{darkblue}{RGB}{5,10,85}
\definecolor{ligthblue}{RGB}{1,147,128}
\definecolor{darkgreen}{RGB}{8,120,51}
\definecolor{darkred}{RGB}{160,0,0}

\lstset{
    language=Scilab,
    captionpos=b,
    extendedchars=true,
    frame=lines,
    numbers=left,
    numberstyle=\tiny,
    numbersep=5pt,
    keepspaces=true,
    breaklines=true,
    showspaces=false,
    showstringspaces=false,
    breakatwhitespace=false,
    stepnumber=1,
    showtabs=false,
    tabsize=3,
    basicstyle=\small\ttfamily,
    backgroundcolor=\color{ligthyellow},
    keywordstyle=\color{ligthblue},
    morekeywords={include, printf, uchar},
    identifierstyle=\color{darkblue},
    commentstyle=\color{darkgreen},
    stringstyle=\color{darkred},
}

%opening
\title{Mise en correspondance stéréoscopique}
\author{Elliot Vanegue}

\begin{document}

\maketitle

\section{Introduction}
Lors de ce TP, nous allons reconstituer la troisième dimension d'une image
qui a été perdu suite à la projection de celle-ci. Pour cela, nous allons 
utiliser la stéréovision afin de reconstruire la troisième dimension de certains points à
partir de deux images dont les angles de vues sont différents.

\section{Calcul de la matrice fondamentale}
La matrice fondamentale contient l'ensemble des informations de la géométrie épipolaire.
Elle est constitué de sept degrès de liberté. Cette géométrie décrit les relations entre
différentes photo du même objet. Pour la calculer, il faut utiliser la formule suivante :
\begin{equation}
 F=(P_2O_1)^xP_2P^+_1
\end{equation}



\end{document}
