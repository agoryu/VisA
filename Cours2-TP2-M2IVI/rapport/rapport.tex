\documentclass[a4paper,11pt]{article}
\usepackage[T1]{fontenc}
\usepackage[utf8]{inputenc}
\usepackage{lmodern}
\usepackage[francais]{babel}

\usepackage{listings}

%couleur
\usepackage{color}
\usepackage{multicol}

%image
\usepackage{graphicx}

\definecolor{ligthyellow}{RGB}{250,247,220}
\definecolor{darkblue}{RGB}{5,10,85}
\definecolor{ligthblue}{RGB}{1,147,128}
\definecolor{darkgreen}{RGB}{8,120,51}
\definecolor{darkred}{RGB}{160,0,0}
\definecolor{univ}{RGB}{177,23,119}
\definecolor{univmodif}{RGB}{148,74,120}
\definecolor{mygray}{RGB}{100,100,100}

\lstset{
    language=Java,
    captionpos=b,
    extendedchars=true,
    frame=lines,
    numbers=left,
    numberstyle=\tiny,
    numbersep=5pt,
    keepspaces=true,
    breaklines=true,
    showspaces=false,
    showstringspaces=false,
    breakatwhitespace=false,
    stepnumber=1,
    showtabs=false,
    tabsize=3,
    basicstyle=\small\ttfamily,
    backgroundcolor=\color{ligthyellow},
    keywordstyle=\color{ligthblue},
    morekeywords={include, printf, uchar},
    identifierstyle=\color{darkblue},
    commentstyle=\color{darkgreen},
    stringstyle=\color{darkred},
}

\title{Logique floue pour la segmentation d'image couleur}
\author{Elliot Vanegue}

\begin{document}

\maketitle
%\tableofcontents

%\begin{abstract}
%\end{abstract}

\section{Introduction}
Nous avons vu dans le TP précédent ce qu'était la logique floue au travers d'exemples ne portant pas 
sur la segmentation d'image. La logique floue permet en effet d'avoir un niveau d'incertitude sur 
l'appartenance d'une donnée à une classe. Dans ce TP, nous allons voir comment appliquer cette méthode
à la segmentation d'image couleur. Par la suite, nous verrons différentes méthodes dérivées du 
principe de la logique floue afin de comparer leur performance. Tout au long de ce TP, nous utilisons
l'image de la Fig. \ref{fig:reference}.

\begin{figure}[!h]
  \begin{center}
    \includegraphics[width=5cm]{6_classes_RGB.png}
    \caption{Image de référence du TP}
    \label{fig:reference}
  \end{center}
\end{figure}

\section{FCM (Fuzzy C-Means)}
Dans un premier temps, nous utilisons la méthode de la logique floue pour segmenter une image couleur. 
Cette méthode est très proche de l'algorithme du K-means. La différence entre les deux algorithmes
est que dans l'algorithme FCM une donnée n'appartient pas à une classe, mais elle a une certaine
probabilité d'y appartenir. Tout comme pour K-means, l'algorithme a besoin d'un certain nombre de
paramètres pour fonctionner. Ces paramètres sont fournis par l'utilisateur.
\begin{itemize}
  \item c : le nombre de classes à segmenter : ce paramètre détermine le nombre de centroïdes à créer.
  \item m : le degré d'appartenance d'une donnée : ce paramètre permet d'avoir un plus ou moins grand écart
  entre les taux d'appartenance aux classes. Plus il sera élevé, plus l'appartenance d'une donnée à une classe
  est forte. Cependant, si m est trop élevé, la correction des erreurs par les étapes suivantes est plus difficile.
  \item seuil : le seuil de stabilité à partir duquel l'algorithme peut s’arrêter.
\end{itemize}
Au début de l'algorithme, les centroïdes sont placés aléatoirement parmi les données. Puis à chaque itération,
ces centoïdes se rapprochent du centre d'un ensemble de données, tout en s'éloignant les uns des autres. La détermination
de la position des centroïdes se fait grâce au calcul (Eq. \ref{centroide}).
\begin{equation}
  \label{centroide}
  \forall i \in \{1, 2, ..., c\} \ v_i = \frac{\sum_{j=1}^{n} u_{ij}^m * x_j}{\sum_{j=1}^{n} u_{ij}^m}
\end{equation}
Dans cette équation, $n$ est le nombre de pixels. La matrice $u$ représente
le degré d'appartenance des pixels pour une classe. Il est possible de calculer cette matrice avec l'équation
(Eq. \ref{membershipDegree}).
\begin{equation}
 \label{membershipDegree}
  u_{ij}=\left[\sum_{k=1}^{c} \left(\frac{d^2(x_j, v_i)}{d^2(x_j, v_k)}\right)^{\frac{2}{m-1}}\right]^{-1}
\end{equation}
Ces deux calculs permettent donc de placer les centroïdes au centre des classes détectées. Il faut maintenant permettre
à l'algorithme de s’arrêter lorsque le seuil de stabilité est atteint. Pour cela, il faut calculer la performance
de l'étape qui a été calculée, c'est-à-dire qu'il faut minimiser le taux d'appartenance des pixels par la distance
avec le centroïde (Eq. \ref{jp}) et comparer ce résultat avec celui de l'étape précédente (Eq. \ref{jpMin}).

\begin{equation}
  \label{jp}
  J_{FCM}(P)=\sum_{i=1}^{c}\sum_{j=1}^{n}[u_{ij}]^m * d_{ij}^2
\end{equation}

\begin{equation}
  \label{jpMin} 
  J_{FCM}(P) - J_{FCM}(P-1) < seuil
\end{equation}

Nous obtenons la segmentation de la Fig. \ref{fig:segmentationFCM}, qui nous permet d'avoir
un résultat très intéressant. On peut voir sur la courbe représentant $J_{FCM}(P)$ (Fig. \ref{fig:graphFCM}), que l'algorithme
fonctionne en très peu d'étapes.

\begin{figure}[!h]
  \begin{center}
    \includegraphics[width=5cm]{resultat/FCM.png}
    \caption{Résultat de la segmentation d'une image couleur avec l'algorithme FCM}
    \label{fig:segmentationFCM}
  \end{center}
\end{figure}

\begin{figure}[!h]
  \begin{center}
    \includegraphics[width=7cm]{resultat/FCM_graph.png}
    \caption{Graphique de $J_{FCM}(P)$}
    \label{fig:graphFCM}
  \end{center}
\end{figure}

\section{HCM (Hard C-Means)}
L'algorithme HCM est similaire à l'algorithme K-means. Il n'utilise pas de degré d'appartenance
d'une donnée à une classe, soit la donnée appartient à une classe, soit elle n'y appartient pas.
Le changement majeur dans l'algorithme du FCM est donc le calcul de $u_{ij}$. Pour déterminer à quelle
classe appartient une donnée, on compare avec la distance de sa coordonnée avec chaque centroïde et
on choisit la classe ayant la plus courte distance. Cela nous fournit le résultat de la Fig. \ref{fig:hcm}.\\

\begin{figure}[!h]
  \begin{center}
    \includegraphics[width=5cm]{resultat/HCM.png}
    \caption{Résultat de la segmentation d'une image couleur avec l'algorithme HCM}
    \label{fig:hcm}
  \end{center}
\end{figure}

\begin{figure}[!h]
  \begin{center}
    \includegraphics[width=7cm]{resultat/HCM_graph.png}
    \caption{Graphique de $J_{HCM}(P)$}
    \label{fig:graphHCM}
  \end{center}
\end{figure}

Encore une fois, la segmentation est efficace et on voit que $J_{HCM}(P)$ fournit des résultats dès la première
itétation. Cependant, l'algorithme met plus de temps
à trouver la solution qu'avec FCM.
\newpage
\section{PCM (Possibilistic C-Means)}
L'algorithme PCM fonctionne sur le même principe que le FCM tout en introduisant une valeur de pénalité
sur l'appartenance d'une donnée à une classe. Cette pénalité exclut les points trop éloignés de la classe. 
Le calcul de cette pénalité est donné dans l'Eq. \ref{penalite}.

\begin{equation}
  \label{penalite}
  \eta_i = \frac{\sum_{j=1}^{n} u_{ij}^m * x_j}{\sum_{j=1}^{n} u_{ij}^m}
\end{equation}

Cette pénalité est appliquée au calcul de $u_ij$ (Eq. \ref{uijPCM}) et à la performance de l'étape courante (Eq. \ref{jPCM}).

\begin{equation}
  \label{uijPCM}
  \eta_i = \frac{1}{1+\left(\frac{d_{ij}^2}{\eta_i}\right)^{\frac{1}{m-1}}}
\end{equation}
\begin{equation}
  \label{jPCM}
  J_{FCM}(P)=\sum_{i=1}^{c}\sum_{j=1}^{n}[u_{ij}]^m * d_{ij}^2+\sum_{i=1}^{c}\eta_i \sum_{j=1}^{n}[1 - u_{ij}]^m
\end{equation}

\begin{figure}[!h]
  \begin{center}
    \includegraphics[width=5cm]{resultat/PCM.png}
    \caption{Résultat de la segmentation d'une image couleur avec l'algorithme PCM}
    \label{fig:pcm}
  \end{center}
\end{figure}

\begin{figure}[!h]
  \begin{center}
    \includegraphics[width=7cm]{resultat/PCM_graph.png}
    \caption{Graphique de $J_{PCM}(P)$}
    \label{fig:graphPCM}
  \end{center}
\end{figure}

Le résultat de l'algorithme est le même, tout en obtenant des résultat pour $J_{PCM}(P)$ dès les premières itération.
Cependant, cet algorithme est un peu plus lent que HCM.

\section{Algorithme de Davé}
L'algorithme de Davé fonctionne a peu près de la même façon que l'algorithme PCM, sauf qu'il exclut
les pixels trop éloignés de la classe en considérant que ceux-ci sont du bruit.

\begin{figure}[!h]
  \begin{center}
    \includegraphics[width=5cm]{resultat/DAV.png}
    \caption{Résultat de la segmentation d'une image couleur avec l'algorithme de Davé}
    \label{fig:dave}
  \end{center}
\end{figure}

\begin{figure}[!h]
  \begin{center}
    \includegraphics[width=7cm]{resultat/DAV_graph.png}
    \caption{Graphique de $J_{PCM}(P)$}
    \label{fig:graphDAVE}
  \end{center}
\end{figure}

\newpage
\section{Conclusion}
Comme nous avons pu le voir au travers de ce TP, la logique floue permet d'avoir 
une segmentation équivalente à une méthode binaire, tout en trouvant le résultat 
plus rapidement. De plus, les résultats obtenus avec ce type de méthodes sont très 
intéressants, bien que ces méthodes aient besoin de nombreux paramètres.
  
\end{document}
