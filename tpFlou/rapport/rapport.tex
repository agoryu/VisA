\documentclass[a4paper,11pt]{article}
\usepackage[T1]{fontenc}
\usepackage[utf8]{inputenc}
\usepackage{lmodern}
\usepackage[francais]{babel}
\usepackage{graphicx}
\usepackage{listings}

% pour definir des couleurs
\usepackage{xcolor}

%couleur
\usepackage{color}
\usepackage{multicol}

% code color
\definecolor{ligthyellow}{RGB}{250,247,220}
\definecolor{darkblue}{RGB}{5,10,85}
\definecolor{ligthblue}{RGB}{1,147,128}
\definecolor{darkgreen}{RGB}{8,120,51}
\definecolor{darkred}{RGB}{160,0,0}

\lstset{
    language=Python,
    captionpos=b,
    extendedchars=true,
    frame=lines,
    numbers=left,
    numberstyle=\tiny,
    numbersep=5pt,
    keepspaces=true,
    breaklines=true,
    showspaces=false,
    showstringspaces=false,
    breakatwhitespace=false,
    stepnumber=1,
    showtabs=false,
    tabsize=3,
    basicstyle=\small\ttfamily,
    backgroundcolor=\color{ligthyellow},
    keywordstyle=\color{ligthblue},
    morekeywords={include, printf, uchar},
    identifierstyle=\color{darkblue},
    commentstyle=\color{darkgreen},
    stringstyle=\color{darkred},
}

\title{Approche de la logique floue}
\author{Elliot Vanegue}

\begin{document}

\maketitle

\section{Introduction}
Lors de ce TP, nous allons nous familiariser avec les concepts principaux de la logique floue.
La logique floue repose sur un système de proposition dont la valeur est dans l'intervalle [0,1] contrairement aux logique
modales qui reposent sur un système binaire. 

\section{Fonction d'appartenance}
Dans un premier temps, nous programmons trois ensembles flous représentant des intervalles de température.
Pour cela nous définissons une fonction par intervalle afin de définir sa courbe. Dans ces fonctions, nous 
distinguons trois cas possible :
\begin{itemize}
  \item Un intervalle où l'ensemble est vrai (ordonnée = 1)
  \item Un intervalle où l'ensemble est faux (ordonnée = 0)
  \item Un intervalle où l'ensemble a un degré de liberté entre 0 et 1
\end{itemize}
Nous obtenons ainsi le graphique de la Fig. \ref{fig:GraphiqueFlou} 

\begin{figure}[!h]
  \begin{center}
    \includegraphics[width=6cm]{tempFlou.png}
    \caption{Graphique de la partition floue de l'exemple \og Température \fg}
    \label{fig:GraphiqueFlou}
  \end{center}
\end{figure}

\begin{figure}
  \begin{lstlisting}[caption=Fonction de l'ensemble température basse]
    def CalcTempB(i):
      if(i < 10):
          return 1.0
      elif(i<20):
          return 1.0 - ((1.0/10.0) * (i-10.0))
      else:
          return 0.0
  \end{lstlisting}
\end{figure}

Cela nous permet ainsi de déterminer les degrés d'appartenance de la température $16\deg$ pour chaque sous 
ensemble (voir Tab. \ref{tab:temp16}).
\begin{table}[!h]
  \label{tab:temp16}

  \begin{center}
    \begin{tabular}{|c|c|c|}
      \hline
       classe & degré d'appartenance\\
       \hline
       basse & 0.4\\
       moyenne & 0.6\\
       élevé & 0.0\\
    \end{tabular}
    \caption{Résultat du degré d'appartenance de la température 16 dans chaque ensemble}
  \end{center}
\end{table}

\section{Opérateurs de la logique floue}
Il est possible de définir des opérateurs permettant de récupérer en sorti les valeurs maximum ou 
minimum de deux ensembles. Pour cela nous créons les fonctions suivantes :
\begin{figure}[!h]
  \begin{lstlisting}[caption=Opérateur min]
   def Opmin(tab1, tab2):
    size = min(len(tab1), len(tab2))
    size = range(0, size)
    result = []
    for i in size:
        step = [tab1[i], tab2[i]]
        result.append(min(step))

    return result
  \end{lstlisting}
\end{figure}

\begin{figure}[!h]
  \begin{lstlisting}[caption=Opérateur max]
   def Opmax(tab1, tab2):
    size = min(len(tab1), len(tab2))
    size = range(0, size)
    result = []
    for i in size:
        step = [tab1[i], tab2[i]]
        result.append(max(step))

    return result
  \end{lstlisting}
\end{figure}

\begin{figure}[!h]
  \begin{center}
    \includegraphics[width=7cm]{operateurFlou.png}
    \caption{Graphique des fonction min et max de la logique floue}
    \label{fig:operateur}
  \end{center}
\end{figure}
\end{document}
