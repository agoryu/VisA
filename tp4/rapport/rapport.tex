\documentclass[a4paper,10pt]{article}
\usepackage[utf8]{inputenc}

%pour les equations
\usepackage{amsmath}
\usepackage{amssymb}
\usepackage{amsfonts}

%pour les images
\usepackage{graphicx}

% pour definir des couleurs
\usepackage{xcolor}

% pour inclure du code
\usepackage{listings}

% code color
\definecolor{ligthyellow}{RGB}{250,247,220}
\definecolor{darkblue}{RGB}{5,10,85}
\definecolor{ligthblue}{RGB}{1,147,128}
\definecolor{darkgreen}{RGB}{8,120,51}
\definecolor{darkred}{RGB}{160,0,0}

\lstset{
    language=C++,
    captionpos=b,
    extendedchars=true,
    frame=lines,
    numbers=left,
    numberstyle=\tiny,
    numbersep=5pt,
    keepspaces=true,
    breaklines=true,
    showspaces=false,
    showstringspaces=false,
    breakatwhitespace=false,
    stepnumber=1,
    showtabs=false,
    tabsize=3,
    basicstyle=\small\ttfamily,
    backgroundcolor=\color{ligthyellow},
    keywordstyle=\color{ligthblue},
    morekeywords={include, printf, uchar},
    identifierstyle=\color{darkblue},
    commentstyle=\color{darkgreen},
    stringstyle=\color{darkred},
}

%opening
\title{Analyse et traitement 3D}
\author{Elliot Vanegue}

\begin{document}

\maketitle

\section{Introduction}
Lors de ce TP, nous avons étudié l'algorithme \textit{Iterative Closest Point}\footnote{ICP} qui est un algorithme de recalage de forme 3D. Ce type de procédé permet entre autre de comparer des modèles 3D ou est encore utilisé pour des applications de réassemblage de surface.

\section{Donnée en entrée}
L'algorithme ICP requiert deux données en entrée pour pouvoir comparer deux nuages de points :
\begin{itemize}
\item Les deux nuages de points.
\item La condition d'arrêt de l'itération.
\end{itemize}

\section{Principe}
Le principe de l'algorithme est de minimiser la distance entre deux nuages de points. L'algorithme fonctionne en quatre étapes que nous allons présenter.

\subsection{Association des points}
On associe les points grâce aux critères du plus proche voisin. Pour cela, il suffit de calculer la distance euclidienne d'un point avec tous les autres points qui font parti du balayage que nous voulons comparer et de prendre la distance la plus petite.
\begin{equation}
d(\vec{p}, A) = min_{i\in{(1...N)}} d(\vec{p},\vec{a_i})
\end{equation}
Ici p est le point dont on veut minimiser la distance et A est l'ensemble des points à tester.

\subsection{Calcul de la meilleure transformation}
On estime la transformation des points grâce à une fonction d'erreur quadratique moyenne, permettant ainsi de trouver la meilleure transformation possible. Il faut donc trouver les matrices R (matrice de rotation) et T (matrice de translation) afin de minimiser :

\begin{equation} 
 f(\vec{q} = \frac{1}{N_p}\sum_{i=1}^{N_p}||\vec{x_i} - R(\vec{q_R}) * \vec{p_i} - \vec{q_T}||^2
\end{equation}

Ici les éléments à déterminer sont $\vec{q_R}$ et $\vec{q_T}$. Nous pouvons calculer $\vec{q_T}$ avec la formule :

\begin{equation} 
 \vec{q_T} = \vec{\mu_x} - R(\vec{q_R})*\vec{\mu_p}
\end{equation}

Dans cette équation $\vec{\mu_x}$ et $\vec{\mu_p}$ sont les centres de masse des
nuages de points que nous traitons. C'est grâce à ce calcul qu'il est possible de
valider le théorème suivant : \\

\fbox{
\begin{minipage}{1\textwidth}
     \begin{center}
          Si (R,t) est la transformation optimale alors S={$s_t$} et T={$t_c$}\\ 
			ont le même centre de masse.
     \end{center}
\end{minipage}
}\\

Et maintenant il faut calculer la rotation $\vec{q_R}$ :
\begin{equation} 
 \vec{q_R} = UV^T
\end{equation} 
Ou U et V sont des matrices unitaires de taille 3x3.
\subsection{Transformation et itération}
 On effectue la transformation qui a été déterminée dans l'étape précédente.
 Puis on réitere depuis l'étape 2 jusqu'à atteindre le critère de fin d'itération.

\section{Conclusion}
Nous avons donc vu comment fonctionne l'algorithme ICP. Cet algorithme a de nombreuses améliorations, notamment au niveau du calcul de la distance de la première étape de ICP.
Cet algorithme est encore utilisé dans des techniques actuelles comme dans des méthodes
de modélisation d'objet déformable non rigide.
\end{document}
